% Von http://www.noch-mehr-davon.de/aprak.shtml geklaut!
% Festlegen des Dokumententyps
\documentclass[11pt,a4paper,twoside]{article}

% Papierformat
\usepackage{a4}

% Deutsche Sprache (Silbentrennung, usw.)
\usepackage[ngerman]{babel}

% Schrifteneinstellungen
\usepackage{lmodern}
\usepackage[T1]{fontenc}
\usepackage{textcomp}

% bessere Matheunterstützung
\usepackage{amsfonts}
\usepackage{amstext}
\usepackage{amsmath}

% Kodierung
\usepackage{ucs}
\usepackage[utf8x]{inputenc}

% Grafiken einbinden
\usepackage{graphicx}
\usepackage{color}

% Floats inside Floats
\usepackage{subfig}

\usepackage[top=2cm, left=3cm, right=3cm, bottom=2cm]{geometry}
% to change the page dimensions
% this gives us a bit more space

\geometry{a4paper} % or letterpaper (US) or a5paper or....
% \geometry{margin=2in} % for example, change the margins to 2 inches all round
% \geometry{landscape} % set up the page for landscape
%   read geometry.pdf for detailed page layout information

%%% PACKAGES
\usepackage{booktabs} % for much better looking tables

% Proper units
\usepackage{units}

%%% HEADERS & FOOTERS
\usepackage{fancyhdr} % This should be set AFTER setting up the page geometry
\pagestyle{fancy} % options: empty , plain , fancy
\renewcommand{\headrulewidth}{0pt} % customise the layout...
\lhead{}\chead{}\rhead{}
\lfoot{}\cfoot{\thepage}\rfoot{}

%%% SECTION TITLE APPEARANCE
\usepackage{sectsty}
\allsectionsfont{\sffamily\mdseries\upshape} % (See the fntguide.pdf for font help)
% (This matches ConTeXt defaults)

%%% Modifications
\renewcommand{\arraystretch}{1.5} % Give tabular cells more vertical space

% Einfach am Ende jeder Seite
% \formular
%und eventuell \newpage
% schreiben - praktisch!

\newcommand{\formular}{
\begin{table}[!h]
\subfloat{
\begin{tabular}{ll}
Praktikanten: &Julius Strake\\
              &Niklas Bölter\\
Gruppe: & 6 \\
\end{tabular}
}
\subfloat{\hspace{2cm}}
\subfloat{
\begin{tabular}{ll}
Betreuer: & Johannes Schmidt \\
Datum: & 14.09.2012\\
Unterschrift: &\hspace{3cm} \\
\cline{2-2}
\end{tabular}
}
\end{table}
}

%%% END Article customizations
\begin{document}
\section*{Messprotokoll}
\subsection*{A-Praktikum \\ Versuch 12 - Die spezifische Elektronenladung $\nicefrac{e}{m_e}$}


\begin{table}[!ht]
  \caption*{Gerätespezifische Angaben}
  \begin{center}
  \begin{tabular}{| r | c | c || r | c | c |}
    \hline
    \hspace{0.6cm} [\hspace{1.0cm}] & \hspace{2.2cm} & \hspace{2.2cm} & \hspace{0.6cm} [\hspace{1.0cm}] & \hspace{2.2cm} & \hspace{2.2cm} \\
    \hline
     [\hspace{1.0cm}] & \hspace{2.2cm} & \hspace{2.2cm} & [\hspace{1.0cm}] & \hspace{2.2cm} & \hspace{2.2cm} \\
    \hline
     [\hspace{1.0cm}] & \hspace{2.2cm} & \hspace{2.2cm} & [\hspace{1.0cm}] & \hspace{2.2cm} & \hspace{2.2cm} \\
    \hline
  \end{tabular}
  \end{center}
\end{table}


\subsubsection*{Messung 1 - Durchmesser der Elektronenkreisbahn}

\begin{table}[!h]
\begin{tabular}{|r|c|c|c|c|c|}
\hline
$U_B \; [\hspace{1cm}]$ & \hspace{2cm} & \hspace{2cm} & \hspace{2cm} & \hspace{2cm} & \hspace{2cm} \\
\hline
$I \; [\hspace{1cm}]$ & \hspace{2cm} & \hspace{2cm} & \hspace{2cm} & \hspace{2cm} & \hspace{2cm} \\
\hline
$d(U_B, I) \; [\hspace{1cm}]$ & \hspace{2cm} & \hspace{2cm} & \hspace{2cm} & \hspace{2cm} & \hspace{2cm} \\
\hline
\hline
$U_B \; [\hspace{1cm}]$ & \hspace{2cm} & \hspace{2cm} & \hspace{2cm} & \hspace{2cm} & \hspace{2cm} \\
\hline
$I \; [\hspace{1cm}]$ & \hspace{2cm} & \hspace{2cm} & \hspace{2cm} & \hspace{2cm} & \hspace{2cm} \\
\hline
$d(U_B, I) \; [\hspace{1cm}]$ & \hspace{2cm} & \hspace{2cm} & \hspace{2cm} & \hspace{2cm} & \hspace{2cm} \\
\hline
\hline
$U_B \; [\hspace{1cm}]$ & \hspace{2cm} & \hspace{2cm} & \hspace{2cm} & \hspace{2cm} & \hspace{2cm} \\
\hline
$I \; [\hspace{1cm}]$ & \hspace{2cm} & \hspace{2cm} & \hspace{2cm} & \hspace{2cm} & \hspace{2cm} \\
\hline
$d(U_B, I) \; [\hspace{1cm}]$ & \hspace{2cm} & \hspace{2cm} & \hspace{2cm} & \hspace{2cm} & \hspace{2cm} \\
\hline
\hline
$U_B \; [\hspace{1cm}]$ & \hspace{2cm} & \hspace{2cm} & \hspace{2cm} & \hspace{2cm} & \hspace{2cm} \\
\hline
$I \; [\hspace{1cm}]$ & \hspace{2cm} & \hspace{2cm} & \hspace{2cm} & \hspace{2cm} & \hspace{2cm} \\
\hline
$d(U_B, I) \; [\hspace{1cm}]$ & \hspace{2cm} & \hspace{2cm} & \hspace{2cm} & \hspace{2cm} & \hspace{2cm} \\
\hline
\hline
$U_B \; [\hspace{1cm}]$ & \hspace{2cm} & \hspace{2cm} & \hspace{2cm} & \hspace{2cm} & \hspace{2cm} \\
\hline
$I \; [\hspace{1cm}]$ & \hspace{2cm} & \hspace{2cm} & \hspace{2cm} & \hspace{2cm} & \hspace{2cm} \\
\hline
$d(U_B, I) \; [\hspace{1cm}]$ & \hspace{2cm} & \hspace{2cm} & \hspace{2cm} & \hspace{2cm} & \hspace{2cm} \\
\hline
\hline
$U_B \; [\hspace{1cm}]$ & \hspace{2cm} & \hspace{2cm} & \hspace{2cm} & \hspace{2cm} & \hspace{2cm} \\
\hline
$I \; [\hspace{1cm}]$ & \hspace{2cm} & \hspace{2cm} & \hspace{2cm} & \hspace{2cm} & \hspace{2cm} \\
\hline
$d(U_B, I) \; [\hspace{1cm}]$ & \hspace{2cm} & \hspace{2cm} & \hspace{2cm} & \hspace{2cm} & \hspace{2cm} \\
\hline
\end{tabular}
\end{table}
\formular


\end{document}
