\subsection{Spezifische Elektronenladung}
In diesem Versuch wird mit Hilfe von \mensch{Helmholtz}-Spulen ein homogenes Magnetfeld erzeugt. Für den magnetischen Fluss $\vec B$ in der Mitte zwischen den Spulen gilt dann: \citep[S.\,91]{Demtroeder-Exp2}
\begin{align}
  B = \frac{8 \mu_0 N I}{\sqrt{125} R}
  \label{eq:B-Feld}
\end{align}
Dabei ist $R$ der Abstand der beiden Spulen sowie ihr Radius (\mensch{Helmholtz}-Bedingung), N ihre Windungszahl und $I$ der durch die Spulen fließende Strom.

Die Elektronen werden durch die Anodenspannung $U$ beschleunigt, haben danach also die kinetische Energie $E = eU$.
\begin{align*}
  & \frac{1}{2} m_e v^2 = e U \\ 
 \Rightarrow \quad & v = \sqrt{\frac{e}{m_e} 2 U}
\end{align*}
Zusätzlich werden sie mit dem auf negativem Potential befindlichen \mensch{Wehnelt}-Zylinder fokussiert.

Danach werden sie im Magnetfeld der \mensch{Helmholtz}-Spulen auf eine Kreisbahn gelenkt. Diese wirkt also als Zentripetalkraft $F_z = \frac{m v^2}{r}$. Gleichsetzen ergibt:
\begin{align}
  \frac{m_e v^2}{r} &= e v B \nonumber\\
  v^2 &= \left(r \frac{e}{m_e} B\right)^2 \nonumber\\
  \frac{e}{m_e} 2 U &= r^2 \frac{e^2}{m_e^2}B^2 \nonumber\\
  \frac{e}{m_e} &= \frac{2U}{r^2 B^2}. 
  \label{eq:SpezLad}
\end{align}
Dabei ist $r$ der Radius der von den Elektronen beschriebenen Kreisbahn.
