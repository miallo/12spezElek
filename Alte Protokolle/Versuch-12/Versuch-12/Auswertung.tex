Die Bestimmung des Wertes für die spezifische Elektronenladung aus den einzelnen Wertetripeln per Formel~\ref{eq:SpezLad} ergibt die in Tabelle~\ref{tab:Feldstaerke} angegebenen Ergebnisse.

Aus diesen lässt sich nun ein gewichteter Mittelwert berechnen:
\begin{empheq}[box=\fbox]{align*}
  \frac{e}{m_e} &= (1.631 \pm 0.018) \cdot 10^{11} \unitfrac{C}{kg}
\end{empheq}
Die Abweichung vom Literaturwert beträgt etwa $7.3\,\%$.

Zur Fehlerberechnung wurde die Abweichung des zur Strommessung verwendeten Multimeters\footnote{M2012, Fehler im Messbereich: $\pm\unit[21]{mA}$~\citep[S.\,34f]{Anleitung}}, die des zur Spannungsmessung verwendeten Multimeters\footnote{Metramax 12, Fehler im Messbereich: $\pm \unit[2.6]{V}$~\citep[S.\,14]{Metramax}} und eine Abschätzung des Fehlers im Durchmesser genutzt.
Da der Elektronenstrahl durchschnittlich etwa eine Breite von $\unit[1]{mm}$ hatte und die Messung des Nullpunktsversatzes nach Ermessen der Praktikanten etwa die gleiche Ungenauigkeit aufwies, wurde der Fehler auf $\pm \unit[2.5]{mm}$ geschätzt, wobei die zusätzlichen $\unit[0.5]{mm}$ die Ungenauigkeit der Skala widerspiegeln.

Nun wurde die spezifische Elektronenladung, die für den kleinsten Durchmesser errechnet wurde, als exakt angenommen. Es wurde der kleinste Durchmesser gewählt, weil hier die geringsten Abweichungen vom theoretischen Wert des Magnetfelds zu erwarten sind, welcher eigentlich nur für die Symmetrieachse des Spulenaufbaus exakt ist. Hiermit lässt sich nun umgekehrt das Magnetfeld auf der Elektronenkreisbahn berechnen, indem Formel~\ref{eq:SpezLad} umgestellt wird:
\begin{align*}
  B &= \sqrt{\frac{2 U m_\mathrm{e}}{r^2 e}} = \unit[(1.27 \pm 0.02)\,10^{-3}]{T}
\end{align*}
Die Abweichung vom theoretischen Wert beträgt $\unit[0.9]{\%}$.
