Nach dem Aufbau der Versuchsschaltung wird der Raum abgedunkelt und mit Hilfe des in Abbildung~\ref{fig:Aufbau} dargestellten Okulars die Lage des Austrittspunktes der Elektronen aus der Elektronenkanone bestimmt. 
\begin{figure}[htb]
\begin{center}
\def\svgwidth{8cm}
\input{Aufbau.pdf_tex}
\end{center}
\caption{Versuchsaufbau ohne Spannungsquelle. \citep{LP-Spulen}}
\label{fig:Aufbau}
\end{figure}
Anschließend sollte man durch grobes Verändern der Anodenspannung und des Stroms durch die Helmholtz-Spulen überprüfen, in welchen Bereichen eine sinnvolle Messung des Elektronenstrahlradius möglich ist.
Jetzt werden für zwei feste Anodenspannungen je etwa acht Werte bei variiertem Spulenstrom aufgenommen, ebenso für zwei feste Spulenströme bei variierter Anodenspannung.
